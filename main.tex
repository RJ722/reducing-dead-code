% Copyright 2004 by Till Tantau <tantau@users.sourceforge.net>.
%
% In principle, this file can be redistributed and/or modified under
% the terms of the GNU Public License, version 2.
%
% However, this file is supposed to be a template to be modified
% for your own needs. For this reason, if you use this file as a
% template and not specifically distribute it as part of a another
% package/program, I grant the extra permission to freely copy and
% modify this file as you see fit and even to delete this copyright
% notice. 

\documentclass{beamer}

% There are many different themes available for Beamer. A comprehensive
% list with examples is given here:
% http://deic.uab.es/~iblanes/beamer_gallery/index_by_theme.html
% You can uncomment the themes below if you would like to use a different
% one:
%\usetheme{AnnArbor}
%\usetheme{Antibes}
%\usetheme{Bergen}
%\usetheme{Berkeley}
%\usetheme{Berlin}
%\usetheme{Boadilla}
%\usetheme{boxes}
%\usetheme{CambridgeUS}
%\usetheme{Copenhagen}
%\usetheme{Darmstadt}
%\usetheme{default}
%\usetheme{Frankfurt}
%\usetheme{Goettingen}
%\usetheme{Hannover}
%\usetheme{Ilmenau}
%\usetheme{JuanLesPins}
%\usetheme{Luebeck}
\usetheme{Madrid}
%\usetheme{Malmoe}
%\usetheme{Marburg}
%\usetheme{Montpellier}
%\usetheme{PaloAlto}
%\usetheme{Pittsburgh}
%\usetheme{Rochester}
%\usetheme{Singapore}
%\usetheme{Szeged}
% \usetheme{Warsaw}

\title{Reducing dead code ratio of Python projects}

% A subtitle is optional and this may be deleted
% \subtitle{Optional Subtitle}

\author{Rahul Jha}
% - Give the names in the same order as the appear in the paper.
% - Use the \inst{?} command only if the authors have different
%   affiliation.

% \institute[Universities of Somewhere and Elsewhere] % (optional, but mostly needed)
% {
%   \inst{1}%
%   Department of Computer Science\\
%   University of Somewhere
%   \and
%   \inst{2}%
%   Department of Theoretical Philosophy\\
%   University of Elsewhere}
% - Use the \inst command only if there are several affiliations.
% - Keep it simple, no one is interested in your street address.

\date{PyCon India, 2017}
% - Either use conference name or its abbreviation.
% - Not really informative to the audience, more for people (including
%   yourself) who are reading the slides online

\subject{Reducing dead code ratio}
% This is only inserted into the PDF information catalog. Can be left
% out. 


% Delete this, if you do not want the table of contents to pop up at
% the beginning of each subsection:
\AtBeginSubsection[]
{
  \begin{frame}<beamer>{Outline}
    \tableofcontents[currentsection,currentsubsection]
  \end{frame}
}

% import minted for code snippets
\usepackage{minted}

% Let's get started
\begin{document}

\begin{frame}
  \titlepage
\end{frame}

\begin{frame}{Outline}
  \tableofcontents
  % You might wish to add the option [pausesections]
\end{frame}

% Section and subsections will appear in the presentation overview
% and table of contents.
\section{Motivation}

\subsection{What is dead code?}

\begin{frame}{What is dead code?}{Definition}
  In computer programming, dead code is a part of the source code of a program which can never be executed because there exists no control flow path to the code from the rest of the program
\newline
\newline
  Furthermore, we can classify it in two types:
  \begin{itemize}
      \item {
        Unused code
      }
      \item {
        Unreachable code
      }
  \end{itemize}

\end{frame}

\begin{frame}{What is dead code?}{Unused code Examples}
    \begin{example}{Something is defined, but never used}
        \inputminted{python}{unused_code.py}
    \end{example}   
\end{frame}

\begin{frame}{What is dead code?}{Unreachable code Examples}
    \begin{example}{Code after return statements:}
        \inputminted{python}{code_after_return.py}
    \end{example}

    \begin{example}{Unsatisfiable Boolean conditions for if/else/while:}
        \inputminted{python}{unsatisfiable_conditions.py}
    \end{example}
\end{frame}

\subsection{How does dead code crawl in?}
\begin{frame}{How does dead code crawl in?}
    \begin{itemize}
        \item{
            Refactoring
        \pause
        }
        \item{
            During debugging
        \pause
        }
        \item{
            Misspellings
        }
    \end{itemize}    
\end{frame}

\subsection{Why remove dead code?}

% You can reveal the parts of a slide one at a time
% with the \pause command:
\begin{frame}{Why remove dead code?}
  \begin{itemize}
  \item {
    Ensures high level of code quality
    \pause
  }
  \item {
    Shrinks program size.
    \pause % The slide will pause after showing the first item
  }
  \item {   
    Reduces running time \pause and therefore, reduces computation power.
    \pause
  }
  % You can also specify when the content should appear
  % by using <n->:
  \item<4-> {
    Sometimes, bugs are also introduced because of dead code.
  }
  \end{itemize}
\end{frame}

\section{Vulture}

\subsection{What is vulture?}

\begin{frame}{Vulture}{An overview}
Repository: \url{https://github.com/jendrikseipp/vulture}

\begin{itemize}
    \item
    A command line tool developed by \href{https://github.com/jendrikseipp}{Jendrik Seipp}
    \pause
    \item
    It helps you find unused code in Python programs and it is useful for 
    cleaning up and finding errors in large code bases
    \pause
    \item
    If you run Vulture on both your library and test suite you can find untested code
    \pause
    \item
    Also, code that is only called implicitly may be reported as unused
\end{itemize}
\end{frame}

\subsection{How to use vulture?}

\begin{frame}{How to use vulture?}{A quick tutorial}

\begin{itemize}
    \item{
        vulture can be installed using pip. It supports both Python2 and 3.
    }
    \item{
        For system wide installation, command:
        \begin{verbatim}

            pip install vulture --user

        \end{verbatim}
    }
    \item{
        Run vulture
        \begin{verbatim}

            vulture myproject/

        \end{verbatim}
    }
\end{itemize}
\end{frame}

\begin{frame}{Additional arguments for vulture}{--sort-by-size}

--sort-by-size \begin{math} \implies \end{math} Sort the results according to the number of lines of code.

\begin{example}
    \begin{verbatim}

        vulture myproject/ --sort-by-size

    \end{verbatim}

\end{example}
\end{frame}

\begin{frame}{Additional arguments for vulture}{--exclude}
    
--exclude \begin{math} \implies \end{math} Comma-separated list of paths to ignore 

\begin{itemize}
    \item PATTERNs can contain globbing
    characters (*, ?, [, ]).
    \item Treat PATTERNs without
    globbing characters as *PATTERN*.
\end{itemize}


\begin{example}
        \begin{verbatim}

        vulture . --exclude=*settings.py,docs/*.py

        \end{verbatim}
        
\end{example}
\end{frame}

\begin{frame}{Additional arguments for vulture}{--min-confidence}
Every result has a confidence assosiated with it.

This flag can be used to specify a minimum confidence which the result must have in order to render the result.

\begin{example}
        \begin{verbatim}

            vulture myproject/ --min-confidence=50

        \end{verbatim}

\end{example}
\end{frame}

\subsection{How to handle false-positives?}

\subsection{How does vulture work?}

\begin{frame}{How does vulture work?}
\end{frame}

\subsection*{vision}
\begin{frame}{Where now?}
    
\end{frame}

\section{Automated analysis through vulture}

\subsection{Adding vulture to CI tests}
\subsection{custom script}
\subsection{VultureBear: An integration with coala}
\subsection{Using GitMate}

% Placing a * after \section means it will not show in the
% outline or table of contents.
\section*{Summary}

\begin{frame}{Summary}
  \begin{itemize}
  \item
    The \alert{first main message} of your talk in one or two lines.
  \item
    The \alert{second main message} of your talk in one or two lines.
  \item
    Perhaps a \alert{third message}, but not more than that.
  \end{itemize}
  
  \begin{itemize}
  \item
    Outlook
    \begin{itemize}
    \item
      Something you haven't solved.
    \item
      Something else you haven't solved.
    \end{itemize}
  \end{itemize}
\end{frame}



% All of the following is optional and typically not needed. 
\appendix
\section<presentation>*{\appendixname}
\subsection<presentation>*{For Further Reading}

\begin{frame}[allowframebreaks]
  \frametitle<presentation>{For Further Reading}
    
  \begin{thebibliography}{10}
    
  \beamertemplatebookbibitems
  % Start with overview books.

  \bibitem{Author1990}
    A.~Author.
    \newblock {\em Handbook of Everything}.
    \newblock Some Press, 1990.
 
    
  \beamertemplatearticlebibitems
  % Followed by interesting articles. Keep the list short. 

  \bibitem{Someone2000}
    S.~Someone.
    \newblock On this and that.
    \newblock {\em Journal of This and That}, 2(1):50--100,
    2000.
  \end{thebibliography}
\end{frame}

\end{document}


